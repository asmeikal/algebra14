\documentclass[11pt,a4paper,twoside]{report}
% draft mode prevents link generation

\usepackage[italian]{babel}
% per date e ToC in italiano
\usepackage{hyperref}
\usepackage{nameref}
\usepackage[table]{xcolor}
\usepackage{amsmath}        % matematica
\usepackage{amsthm}         % teoremi
\usepackage{thmbox}
\usepackage{amssymb}        % simboli
\usepackage{amsfonts}       % font matematici
\usepackage{mathrsfs}
\usepackage{mathtools}
\usepackage{centernot}      % per semplificare

\usepackage{fullpage}       % troppo margine normalmente
\usepackage{parskip}        % preferisco spazio fra i paragrafi
                            % all'indentazione sulla prima riga

\usepackage{pgfplots}       % grafici
\pgfplotsset{compat=1.5}
\usetikzlibrary{shapes,arrows,calc,fit,backgrounds}

\usepackage{fancyhdr}

\pagestyle{fancyplain}
\fancyhead{} % clear all header fields
\fancyfoot{} % clear all footer fields
\fancyfoot[C]{\today}
\fancyfoot[LE,RO]{\thepage}
\renewcommand{\headrulewidth}{0pt}
\renewcommand{\footrulewidth}{0pt}

\theoremstyle{plain}
\newtheorem{axiom}{Assioma}
\newtheorem{theorem}{Teorema}[section]
\newtheorem{lem}[theorem]{Lemma}
\newtheorem{prop}[theorem]{Proposizione}
\newtheorem{cor}[theorem]{Corollario}

\theoremstyle{definition}
\newtheorem{defn}{Definizione}[section]
\newtheorem{esercizio}{Esercizio}

\theoremstyle{remark}
\newtheorem{oss}{Osservazione}
\newtheorem{caso}{Caso}
\newtheorem{fact}{Fatto}

\newenvironment{exmp}[1][]
{ \par\medskip\textbf{Esempio\ #1\unskip\,:}\\*[1pt]\rule{\textwidth}{0.4pt}\\*[1pt] }
{ \\*[1pt]\rule{\textwidth}{0.4pt}\medskip }

\newenvironment{smallpmatrix}
{\left( \begin{smallmatrix}}
{\end{smallmatrix} \right)}

\newenvironment{completepmatrix}[1]
{\left(\begin{array}{*{#1}{c}|c}}
{\end{array}\right)}

\newcommand{\covers}{<\!\cdot}                      % simbolo di 'copre'
\newcommand{\compl}{\mathsf{c}}                     % C complementare
\newcommand{\dotcup}{\mathaccent\cdot\cup}          % unione disgiunta
\newcommand{\lateralsx}[1]{{}_{#1}\!\sim}           % equivalenza sinistra
\newcommand{\lateraldx}[1]{\sim_{#1}}               % equivalenza destra
\newcommand{\mcm}{\operatorname{mcm}}               % mcm
\newcommand{\mcd}{\operatorname{MCD}}               % MCD
\newcommand{\pow}[1]{<\!{#1}\!>}                    % gruppo delle potenze (<a>)
\newcommand{\abs}[1]{\left\lvert{#1}\right\rvert}   % valore assoluto
\newcommand{\norm}[1]{\left\lVert{#1}\right\rVert}  % norma di un vettore
\newcommand{\divides}{\bigm|}                       % simbolo di 'divide'
\newcommand{\definition}{%
    \overset{\underset{\mathrm{def}}{}}{=}}         % uguale con 'definizione' sopra
\newcommand{\supop}{\vee}                           % simbolo dell'operatore sup
\newcommand{\infop}{\wedge}                         % simbolo dell'operatore inf
\newcommand{\subgroupset}{\mathscr{S}}              % insieme dei sottogruppi (S corsiva)
\newcommand{\solutions}{\mathscr{S}}                % insieme delle soluzioni
\newcommand{\parts}{\mathbb{P}}                     % insieme dei naturali
\newcommand{\naturals}{\mathbb{N}}                  % insieme dei naturali
\newcommand{\integers}{\mathbb{Z}}                  % insieme degli interi
\newcommand{\rationals}{\mathbb{Q}}                 % insieme dei razionali
\newcommand{\reals}{\mathbb{R}}                     % insieme dei reali
\newcommand{\complexes}{\mathbb{C}}                 % insieme dei complessi
\newcommand{\field}{\mathbb{K}}                     % simbolo di campo generico
\newcommand{\subfield}{\mathbb{F}}                  % simbolo di sottocampo generico
\newcommand{\matrices}{\mathfrak{M}}                % insieme delle matrici
\newcommand{\nullelement}{\underline{0}}            % elemento neutro
\newcommand{\image}[1]{Im_{#1}}                     % immagine
\newcommand{\detname}{\operatorname{det}}           % determinante
\renewcommand{\det}[1]{\detname \left({#1}\right)}  % determinante(A)
\newcommand{\algcompl}[2][A]{{\mathcal{#1}}_{#2}}   % complemento algebrico
\newcommand{\agg}[1]{\operatorname{Agg}%
    \left( {#1} \right)}                            % matrice aggiunta
\newcommand{\id}{id}                                % identita'
\newcommand{\sfrac}[2]{{#1}/{#2}}                   % frazione su una riga sola
\newcommand{\requiv}[1][]{\ \varepsilon_{#1} \ }    % relazione di equivalenza

% maggiore semantica
\let\oldforall\forall
\let\oldexists\exists
\renewcommand{\forall}{\ \oldforall \ }
\renewcommand{\exists}{\ \oldexists \ }
\renewcommand{\implies}{\Rightarrow}
\renewcommand{\iff}{\Leftrightarrow}

\renewcommand{\labelitemi}{$-$}

\makeatletter
\let\orgdescriptionlabel\descriptionlabel\renewcommand*{\descriptionlabel}[1]{%
\let\orglabel\label
\let\label\@gobble
\phantomsection
\edef\@currentlabel{#1}%
%\edef\@currentlabelname{#1}%
\let\label\orglabel
\orgdescriptionlabel{#1}%
}
\makeatother

\author{Michele Laurenti \\ \href{mailto:asmeikal@me.com}{asmeikal@me.com}}
\date{\today}
