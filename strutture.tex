% WEB

\begin{center}
\indent
\textit{Gruppi, anelli, campi. In particolare, anello degli interi modulo $m$ intero, anello dei polinomi.}
\end{center}

\section{Strutture algebriche con un'operazione}

Una struttura algebrica \`e una coppia $(A, \cdot)$ dove A \`e un'insieme e $\cdot$ \`e un'operazione $\cdot : A \times A \to A$. Ad esempio $(\mathbb{N}, +)$ \`e una struttura algebrica.

Le operazioni sono funzioni definite su prodotti cartesiani a valori in un insieme. Un'operazione binaria \`e definita sul prodotto cartesiano fra due insiemi.

Riprendendo la composizione, dati tre insiemi $A, B, C$, $B^A$ \`e l'insieme delle funzioni da $A$ in $B$, $C^B$ \`e l'insieme delle funzioni da $B$ in $C$. La composizione $\circ$ \`e un'operazione definita sul prodotto cartesiano degli insiemi $B^A \times C^B \to C^A$.

Posso rappresentare un'operazione come funzione $(\circ \left( f, g \right))$ o inserendo l'operatore fra i due operandi $ (g \circ f) $.
\begin{gather*}
f: \mathbb{R} \times \mathbb{R} \to \mathbb{R} ; \
f(x,y) = \sqrt{2} \cdot x + y \\
g: \mathbb{R} \to \mathbb{R} \times \mathbb{R} ; \
g(z) = (0,z) \\
g \circ f : \mathbb{R} \times \mathbb{R} \to \mathbb{R} \times \mathbb{R} \\
(x,y) \xrightarrow{f} \sqrt{2} \cdot x + y = z \xrightarrow{g} \left( 0, \sqrt{2}x + y \right) \\
f \circ g : \mathbb{R} \to \mathbb{R} \\
z \xrightarrow{g} (0,z) \xrightarrow{f} \sqrt{2} \cdot 0 + z = z
\end{gather*}

Una struttura algebrica \`e un'insieme su cui \`e definita un'operazione che prende due elementi di quell'insieme e gliene associa un terzo.

Le strutture vengono classificate in base alle loro propriet\`a:
\begin{description}
    \item[Propriet\`a associativa\label{itm:strutture_associativa}] $\forall \ a, b, c \in A : a \cdot (b \cdot c) = (a \cdot b) \cdot c$
    \item[Elemento neutro\label{itm:strutture_neutro}] Esistenza di un'elemento neutro, o elemento identit\`a. $1 \in A : \forall \ a \in A , a \cdot 1 = a = 1 \cdot a$
    \item[Propriet\`a commutativa\label{itm:strutture_commutativa}] $ \forall a, b \in A , a \cdot b = b \cdot a $. In una struttura algebrica commutativa in genere l'identit\`a si indica con 0.
    \item[Inverso\label{itm:strutture_inverso}] Esistenza dell'inverso. $ \forall a \in A \ \exists \ b \in A $ t.c. $a \cdot b = 1 = b \cdot a $.
\end{description}

\section{Classificazione delle strutture algebriche con una operazione}

Per essere studiabile, una struttura algebrica deve essere quantomeno associativa.

\begin{description}
    \item[Semigruppo] struttura algebrica associativa.
    \item[Monoide] struttura algebrica associativa con elemento identit\`a.
    \item[Gruppo] struttura algebrica associativa con elemento identit\`a e con inverso (ossia, monoide con inverso).
    \item[Gruppo abeliano] struttura algebrica che presenta tutte e quattro le propriet\`a: associativa, elemento neutro, commutativa, inverso.
\end{description}

La struttura algebrica $\left( \mathbb{N}, + \right)$ \`e un monoide commutativo. Anche $\left( \mathbb{N}, \cdot \right)$ \`e un monoide commutativo. $\left( \mathbb{Z}, + \right)$ \`e un gruppo perch\`e esiste l'inverso. $\left( \mathbb{Z}, \cdot \right)$ \`e un monoide, perch\'e non ha l'inverso.

Il prototipo di tutti i gruppi \`e il gruppo simmetrico su $n$ elementi, il cui insieme \`e indicato con $S_n$. Prendiamo un insieme $E = \left\{ e_1, \dots, e_n \right\}$.

\[
S_n = \left\{ f : E \to E \text{ t.c. $f$ \`e biunivoca} \right\}
\]

Il gruppo simmetrico \`e definito sull'insieme $S_n$ e l'operazione \`e la composizione: $\left( S_n, \circ \right)$.
\begin{enumerate}
    \item $f \circ \left( g \circ h \right) = \left( f \circ g \right) \circ h$ 
    \item L'unit\`a \`e la funzione identica (o identit\`a) $i_E : E \to E$ tale che $\forall \ e \in E $ $i_E(e) = e$. $f \circ i_E = f = i_E \circ f$ 
    \item Una funzione biunivoca $f$ ha una funzione inversa $g$. $g : E \to E $ t.c. $ g(f(e)) = e$.
\end{enumerate}

Una funzione $f : E \to E $ iniettiva su un insieme finito $E$ \`e necessariamente suriettiva e quindi biunivoca.

Un insieme \`e finito se non pu\`o essere messo in corrispondenza biunivoca con un suo sottoinsieme proprio.

% \subsection{Punto di vista dell'occupazione}

% $f : \left \{ 1, \dots, 6 \right \} \to \left \{ 1, \dots, 6 \right \}$. Penso il dominio come degli oggetti. Il codominio come dei ``cassetti''. La funzione \`e un modo di mettere gli oggetti del dominio nei ``cassetti''.

% \begin{tabular}{cccccc}
% 1 & 2 & 3 & 4 & 5 & 6 \\
% 2 & 3 & 5  &1 & 6 & 4
% \end{tabular}
% \`E un'occupazione.

% \begin{tabular}{cccccc}
% 1 & 2 & 3 & 4 & 5 & 6 \\
% 6 & 6 & 3 & 5 & 5 & 5
% \end{tabular}
% Non \`e un'occupazione.

% SONO ARRIVATO QUI

\subsection{Monoidi}

Un monoide $(M, \cdot)$ \`e una struttura algebrica con un'operazione $\cdot : M \times M \to M$ tale che:
\begin{description}
    \item[1M] L'operazione $\cdot$ \`e associativa;
    \item[2M] $\exists 1_M \forall a \in M $ t.c. $1_M \cdot a = a = a \cdot 1_M$, ossia esiste l'elemento unit\`a. 
\end{description}

Un sottomonoide $(S, \cdot)$ con $S \subseteq M$ \`e un onoide in cui $\cdot S \times S \to S$, ossia $S$ \`e chiuso, cio\`e $s \cdot s' \in S$ e $1_M \in S$.

Ad esempio, considerando $(\mathbb{N}, +)$, $(P, +)$ con $P = \{ m \in \mathbb{N} : \exists k $ t.c. $m = 2k \}$ \`e un sottomonoide di $(\mathbb{N},+)$ perch\`e la somma di due pari \`e pari e lo 0 appartiene ai pari.

$k \mathbb{N} = \{ m \in \mathbb{N} : \exists t \in \mathbb{N} $ t.c. $ m = k t\}$ \`e la ``versione generale'' dell'insieme dei numeri pari.

Considerando $(\mathbb{N}, \cdot)$ e l'elemento neutro 1, i pari non sono un sottomonoide, ma i dispari s\`i.

\subsection{Morfismi di monoidi}

\begin{defn}
Dati $(M, \cdot)$ e $(A, \ast)$ un morfismo di monoidi \`e un'applicazione $f : M \to A$ che conserva le strutture, ossia tale che $\forall \ x,y \in M $ ho che $f(x \cdot y) = f(x) \ast f(y)$. Inoltre, $f(1_M) = 1_A$.
\end{defn}

\subsection{Teorema di omomorfismo per i monoidi}

\begin{prop}
Sia $f : (M, \cdot) \to (A, \ast)$ un morfismo di monoidi, $f$ definisce una relazione di equivalenza $\varepsilon_f$ tale che $\ker f = M / \varepsilon_f \cong Im_f$, ossia il quoziente \`e isomorfo all'immagine. Inoltre il quoziente ha una struttura di monoide:
\[
(M / \varepsilon_f , \cdot) \cong (Im_f, \ast)
\]
Con $(Im_f, \ast)$ sottomonoide di $(A, \ast)$.
\end{prop}
\begin{proof}
Ogni morfismo di monoidi individua un sottomonoide di $(A, \ast)$, che \`e $(Im_f, \ast)$. $\forall f(x), f(y) \in Im_f$, $f(x) \ast f(y) = f(x \cdot y) \in Im_f$, quindi $Im_f $ \`e chiuso rispetto a $\ast$. Inoltre $1_A \in Im_f \Rightarrow f(1_M) = 1_A$ perch\'e \`e un morfismo di monoidi.

$[x] = \{ y \in M : x \varepsilon_f y \Leftrightarrow f(x) = f(y) \}$ $(\ker f, \cdot)$ \`e un monoide.

$[x] \cdot [z] = [x \cdot z]$ qualsiasi rappresentante scelgo della classe. Devo quindi verificare che la definizione sia indipendente dai rappresentanti. Inoltre ho l'unit\`a $[1_M]$ in $\ker f$.

L'isomorfismo fra $(Im_f, \ast)$ e $(\ker f, \cdot)$ segue naturalmente dal fatto che $\ker f $ e $Im_f$ sono in biezione, ed inoltre, essendo monoidi, la funzione $f$ \`e un morfismo di monoidi.
\end{proof}

\subsection{Potenze (iterazioni sui monoidi)}

A partire dal monoide $(M, \cdot)$ possiamo definire le iterazioni dell'operazione $\cdot$, ossia le potenze.

Sia $ a \in M$, si definisce:
\begin{itemize}
    \item $a^0 = 1_M$
    \item $a^{n+1} = a \cdot a^{n}$
\end{itemize}
Che propriet\`a hanno le iterazioni? $\forall n \in \mathbb{N}$, $a \cdot a^n = a^n \cdot a$, ossia \`e commutativa. Lo vediamo subito con $n = 0$, $a \cdot a^0 = a = a^0 \cdot a$.

Per definizione $a \cdot a^{n+1} = a \cdot a \cdot a^{n} = a \cdot a^n \cdot a = a^{n+1} \cdot a$ per ipotesi induttiva.

Abbiamo anche le classiche propriet\`a delle potenze:

$a^{m + n} = a^m \cdot a^n$

Si dimostra anche questo per induzione su $n$. Con $n = 0$, $a^{m+0} = a^m = a^m \cdot 1_M = a^m \cdot a^0$.

Passo induttivo. $a^{m + n + 1} = a \cdot a^{m + n}$ per definizione di potenze. Applicando l'ipotesi induttiva, $a \cdot a^{m + n} = a \cdot a^m \cdot a^n$. Per commutativit\`a $a \cdot a^m \cdot a^n = a^m \cdot a \cdot a^n = a^m \cdot a^{n+1}$.

\begin{theorem}
Dato $(M, \cdot)$ monoide ed un elemento $a \in M$, esiste un solo morfismo di monoidi $f : (\mathbb{N}, +) \to (M, \cdot)$ tale che $f(1) = a$.
\end{theorem}
\begin{proof}
$f(n) = a^n$. $f$ \`e un morfismo di monoidi. Infatti verifica $f(m+n) = f(m) \cdot f(n)$ e $f(0) = 1_M$. Per le propriet\`a dimostrare precedentemente, $f(m+ n) = a^{m+n} = a^m \cdot a^n = f(m) \cdot f(n)$, e $f(0) = a^0 = 1_M$ per definizione di potenze. Infine, $f(1) = a^1 = a \cdot a^0 = a \cdot 1_M = a$.

Dimostriamone ora l'unicit\`a. Sia $g : (\mathbb{N}, +) \to (M, \cdot )$ tale che $g(1) = a$, dimostriamo che $\forall n \in \mathbb{N} g(n) = f(n) = a^n$.

Dimostriamolo per induzione su $n$. Per definizione di morfismo di monoidi, $g(0) = 1_M = f(0) = a^0$.

Supponiamo che $g(n) = f(n)$, per definizione di morfismo di monoidi $g(n+1) = g(1) \cdot g(n) = a \cdot g(n) = a \cdot f(n) = a \cdot a^n = a^{n+1}$.
\end{proof}

Sia $\Gamma$ un insieme, prendo $(\mathbb{P}(\Gamma), \cup)$. \`E una struttura algebrica, ed in particolare un monoide. L'unione \`e associativa ($(A \cup B) \cup C = A \cup (B \cup C)$), ed esiste l'elemento neutro $\emptyset$.

Anche $(\mathbb{P}(\Gamma), \cap)$ \`e un monoide, con elemento neutro pari a $\Gamma$, poich\'e $\forall A$ $\Gamma \cap A = A$. Abbiamo quindi due esempi di monoidi commutativi.

Fissiamo $\emptyset \neq S \subseteq \Gamma$, possiamo considerare $\mathbb{P}(S)$ e definire l'applicazione $f : \mathbb{P}(\Gamma) \to \mathbb{P}(S)$ tale che $f(A) = A \cap S$. 

Vediamo se questa applicazione \`e un morfismo di monoidi considerando $(\mathbb{P}(\Gamma), \cup)$ e $(\mathbb{P}(S), \cup)$. Dobbiamo dimostrare che $f( A \cup B) = f(A) \cup f(B)$. Infatti $f(A \cup B) = (A \cup B) \cap S = (A \cap S) \cup (B \cap S)$ per la propriet\`a distributiva, che \`e proprio $f(A) \cup f(B)$.

Inoltre conserva l'elemento neutro, poich\'e $f(\emptyset) = \emptyset \cap S = \emptyset$.

Vediamo se \`e un morfismo di monoidi anche considerando $(\mathbb{P}(\Gamma), \cap)$ e $(\mathbb{P}(S), \cap)$. $f(A \cap B) = f(A) \cap f(B)$, infatti $(A \cap B) \cap S - (A \cap S) \cap (B \cap S)$. Conserva anche l'elemento neutro, poich\'e $f(\Gamma) = \Gamma \cap S = S$.

Sia $f : \mathbb{P}(\Gamma) \to \mathbb{P}(\Gamma)$ tale che $f(A) = \bar A = \{ x \in Gamma : x \notin A\}$. Poich\'e $\bar{A \cup B} = \bar A \cap \bar B$, $f : (\mathbb{P}(\Gamma), \cup) \to (\mathbb{P}(\Gamma), \cap)$ \`e un morfismo di monoidi visto che $f(A \cup B) = f(A) \cap f(B)$, e $f(\emptyset) = \bar \emptyset = \Gamma$. 

Possiamo considerare la stessa applicazione come un morfismo di monoidi da $f : (\mathbb{P}(\Gamma), \cap) \to (\mathbb{P}(\Gamma), \cup)$. Infatti $f(A \cap B) = \bar{A \cap B} = \bar A \cup \bar B = f(A) \cup f(B)$ e $f(\Gamma) = \bar \Gamma = \emptyset$.

\subsection{Congruenze}

Le congruenze sono relazioni d'equivalenza definite sulle strutture algebriche. Sia $\varepsilon$ una relazione d'equivalenza su $A$, con $(A, \cdot)$ monoide, si dice congruenza rispetto a $\cdot$ se, dati $a \varepsilon b$ e $c \varepsilon d$, ho che $a \cdot c \varepsilon b \cdot d$.

Vuol dire che, dati $a \in [a]$ e $c \in [c]$, se $a \cdot c \in [a \cdot c]$ e ho una congruenza, allora $b \in [a]$ e $d \in [c]$ sono tali che $b \cdot d \in [a \cdot c]$. La congruenza fa s\`i che io possa definire le operazioni sulle classi.

Considerando quindi un morfismo di monoidi $f : (M, \cdot) \to (A, \ast)$, se definisco la relazione di equivalenza $\varepsilon_f$ tale che $x, y \in M$ sono $x \varepsilon_f y \Leftrightarrow f(x) = f(y)$, questa relazione di equivalenza \`e una congruenza.
\begin{proof}
$x \varepsilon_f y$, $z \varepsilon_f w \Rightarrow (x \cdot z) \varepsilon_f (y \cdot w)$.

Infatti $f(x \cdot z) = f(x) \ast f(y)$ e $f(y \cdot w) = f(y) \ast f(w)$. 
\end{proof}

Avevamo definito $\rho$ su $\mathbb{N} \times \mathbb{N}$, come $(m, n) \rho (a, b) \Leftrightarrow m+b = n+a$. Quindi a partire da $(M, \cdot) $ possiamo creare altri monoidi $(M^n, \cdot)$, ad esempio su $M^2 = M \times M$ in cui $(x, y) \cdot (z, t) = (x \cdot z, y \cdot t)$.

Ad esempio $(\mathbb{N}, +) \to (\mathbb{N} \times \mathbb{N}, +)$ in cui $(m, n) + (a, b) = (m+a, n+b)$ con l'elemento neutro $(0,0)$.

$\rho$ \`e una congruenza rispetto a + in $\mathbb{N} \times \mathbb{N}$. Inoltre, avendo visto che $\mathbb{N} \times \mathbb{N} / \rho = \mathbb{Z}$, abbiamo che $(\mathbb{Z}, +)$ \`e un monoide.

\subsection{Gruppi}

$(G, \cdot)$ \`e un gruppo se:
\begin{description}
    \item[1G] $(G, \cdot)$ \`e un monoide
    \item[2G] $\forall a \in G \exists b \in G $ tale che $a \cdot b = 1_G$, comunemente indicato con $b = a^{-1}$ e detto inverso di $a$.
\end{description}

Possiamo definire un morfismo di gruppi. Un morfismo conserva strutture e propriet\`a. Deve quindi essere un morfismo di monoidi che manda l'inverso nell'inverso.

Quindi $f : (G, \cdot) \to (G', \ast)$ \`e un morfismo di gruppi se:
\begin{enumerate}
    \item \`e un morfismo di monoidi
    \item $\forall a \in G f(a^{-1}) = (f(a))^{-1}$
\end{enumerate}
Ma queste due cose sono conseguenza di una sola propriet\`a, ossia che il morfismo conserva le operazioni. Infatti se $f(a \cdot b) = f(a) \ast f(b)$ allora sono vere tutte le propriet\`a.

Infatti manda le unit\`a nelle unit\`a: $f(1_G) = f(1_G \cdot 1_G) = f(1_G) \cdot f(1_G)$. Ma essendo monoidi hanno l'inverso, quindi $(f(1_G))^{-1} \cdot f(1_G) = (f(1_G))^{-1} \cdot f(1_G) \cdot f(1_G) \Rightarrow f(1_G) = 1_{G'}$.

Manda gli inversi negli inversi: $f(a^{-1}) = f(a)^{-1}$. Infatti $f(a) \ast f(a^{-1}) = f(a \cdot a^{-1}) = f(1_G) = 1_{G'}$.

Sottogruppi

Partendo da $(G, \cdot)$ e scegliendo $\emptyset \neq S \subseteq G$, un sottogruppo $(S, \cdot)$ deve essere:
\begin{itemize}
    \item Chiuso: $\forall s, s' \in S$, $s \cdot s' \in S$
    \item Deve contenere l'unit\`a: $1_G \in S$. E quindi $S$ \`e un sottomonoide di $G$
    \item Per essere anche un sottogruppo, $S$ deve essere chiuso rispetto agli inversi: $s \in S \Rightarrow s^{-1} \in S$.
\end{itemize}
Condizione necessaria e sufficiente affinch\'e $(S, \cdot)$ con $S \neq \emptyset$ sia un sottogruppo del gruppo $(G, \cdot)$ :
\[
\forall a, b \in S \Rightarrow a^{-1} \cdot  b \in S
\]
Dimostrare che \`e condizione necessaria \`e banale. 

Per definizione di sottogruppo $a^{-1}$ \`e in $S$, ed essendo chiuso $a^{-1} \cdot b \in S$.

Dobbiamo dimostrare che \`e sufficiente. $S \neq \emptyset$, quindi ha almeno un elemento $a \in S$. Prendiamo $b = a^{-1} $, per la propriet\`a sopra $a \cdot a^{-1} \in S \Rightarrow 1_G \in S$.

$\forall x \in S \Rightarrow x^{-1} \in S$ sempre per la propriet\`a sopra. Infatti prendendo $a = x$ e $b = 1_G$, $a^{-1} \cdot b \in S$ quindi $x^{-1} \cdot 1_G \in S \Rightarrow x^{-1} \in S$.

Chiusura: $\forall s, s' \in S \Rightarrow s \cdot s' \in S$. Abbiamo gi\`a che $s \in S \Rightarrow s^{-1} \in S$, quindi per la solita propriet\`a ho che $(s^{-1})^{-1} \cdot s' \in S \Rightarrow s \cdot s' \in S$.

$(\mathbb{Z}, +) \to (\mathbb{N} \times \mathbb{N}, +) / \rho$.

Quali sono i sottogruppi di $(\mathbb{Z}, +)$? In generale sono sottogruppi di $\mathbb{Z}$ tutti i gruppi $(k \mathbb{Z}, +)$, anzi sono gli unici.

$(\mathbb{Z}, \cdot)$ non \`e un gruppo perch\'e non ha l'inverso per ogni elemento.

$(G, \cdot)$ non \`e commutativo.

$(S_n, \circ)$ composizione e insieme delle funzioni iniettive da un insieme con $n$ elementi in s\'e stesso. Non \`e commutativo.

Dati $a, b \in G$, se voglio avere l'inverso del prodotto $(a \cdot b)^{-1}$.

Ho due funzioni $\sigma$ e $\tau$ dal punto di vista dell'occupazione.

\begin{tabular}{cccc}
$\sigma$
1 & 2 & 3 & 4 \\
2 & 3 & 1 & 4
\end{tabular}

\begin{tabular}{cccc}
$\tau$
1 & 2 & 3 & 4 \\
1 & 2 & 4 & 3
\end{tabular}

\begin{tabular}{cccc}
$\sigma \circ \tau$
1 & 2 & 3 & 4 \\
1 & 2 & 4 & 3 \\
2 & 3 & 4 & 1
\end{tabular}

\begin{tabular}{cccc}
$\tau \circ \sigma$
1 & 2 & 3 & 4 \\
2 & 3 & 1 & 4 \\
2 & 4 & 1 & 3
\end{tabular}

Solo se il gruppo \`e commutativo ho che $(a \cdot b)^{-1} = a^{-1} \cdot b^{-1}$. Quindi se voglio trovare l'inverso di $(\sigma \circ \tau) \circ ? = i$ devo usare $\tau^{-1} \circ \sigma^{-1}$.

L'inverso del prodotto \`e il prodotto degli inversi scambiati di posto.
\[
(a \cdot b) \cdot (b^{-1} \cdot a^{-1}) = a \cdot (b \cdot b^{-1}) 
\cdot a^{-1} = a \cdot 1_G \cdot a^{-1} = 1_G
\]

Un'applicazione $f : G \to G'$ \`e un morfismo di gruppi se $\forall a, b \in G$ ho che $f(a \cdot b) = f(a) \ast f(b) \Rightarrow f(1_G) = 1_{G'}$ e $f(a^{-1}) = (f(a))^{-1}$.

Teorema di omomorfismo per i gruppi.

Esempio: $\log : (\mathbb{R}^{+}, \cdot) \to (\mathbb{R}, +)$ \`e un morfismo di gruppi, perch\'e $\log(a \cdot b) = \log(a) + \log(b)$.

Anche la funzione $\exp : \mathbb{R} + \to \mathbb{R}^{+} \cdot$ \`e un morfismo di gruppi.  $\exp(a + b) = \exp(a) \cdot \exp(b)$

Anche l'iterazione della somma \`e un morfismo di gruppi. $f_n : \mathbb{Z} + \to \mathbb{Z} + $ $\forall z \in \mathbb{Z}$ $f_n(z) = n \cdot z$

Ogni morfismo di gruppi $f$ individua due sottogruppi:
\begin{itemize}
    \item $Im_f \subseteq G'$
    \item $\ker f \subseteq G$ $\ker f = \{ u \in G : f(u) = 1_{G'}\}$, quindi $\ker f$ \`e una classe, ossia $\ker f \in G / \varepsilon_f$
\end{itemize}

Con i monoidi avevamo $(M, \cdot)$ una struttura associativa con $1_M$. Il $\ker f$ l'abbiamo chiamato quoziente, ossia:
\[
\ker f = M / \varepsilon_f
\]
Come mai il nucleo adesso \`e una classe? 

\begin{figure}[ht]
\centering
\begin{tikzpicture}
  \node (A) {$A$};
  \node (f) [right of=A, node distance=2cm] {$f$};
  \node (B) [right of=f, node distance=2cm] {$B$};
  \node (a) [below of=A, node distance=1cm] {$a$};
  \node (b) [below of=a, node distance=1cm] {$b$};
  \node (c) [below of=b, node distance=1cm] {$c$};
  \node (d) [below of=c, node distance=1cm] {$d$};
  \node (e) [below of=d, node distance=1cm] {$e$};
  \node (1) [below of=B, node distance=1cm] {$1$};
  \node (2) [below of=1, node distance=1cm] {$2$};
  \node (3) [below of=2, node distance=1cm] {$3$};
  \node (4) [below of=3, node distance=1cm] {$4$};
  \node (5) [below of=4, node distance=1cm] {$5$};
  \node (6) [below of=5, node distance=1cm] {$6$};
  \node (7) [below of=6, node distance=1cm] {$7$};
  \path[->]  (a) edge node {} (3)
            (b) edge node {} (3)
            (c) edge node {} (4)
            (d) edge node {} (4)
            (e) edge node {} (7)
            ;
\end{tikzpicture}
\caption{\label{fig:esempio_funzione}$\ker f = \{ \{a, b \}, \{ c, d \}, \{ e \}\}$ }
\end{figure}

Nel caso in figura \ref{fig:esempio_funzione} $\ker f = \{ \{a, b \}, \{ c, d \}, \{ e \}\}$. Ho \textit{bisogno} di sapere quali classi ci sono per ricostruire la funzione.

Per conoscere la $f$ devo sapere tutti i blocchi della partizione, non posso ricostruire gli altri blocchi da un blocco solo.

Con i gruppi non \`e cos\`i. Mi basta la classe degli elementi che vanno nell'unit\`a. Se conosco una classe le conosco tutte.

Fissato il $\ker f$ conosco tutti gli elementi che hanno la stessa immagine di $a$, ossia $a \cdot \ker f = [a]$.

Abbiamo $f$ morfismo di gruppi $ f : (G, \cdot ) \to (G', \ast)$. Dimostriamo intanto che il nucleo \`e un sottospazio, ossia un sottogruppo.

$\ker f = [1_G]$ conosco tutti gli elementi che finiscono nell'unit\`a di $G'$

CNES afficnhe un sottoinsieme sia un sottogruppo \`e che se $a, b \in S \Rightarrow a^{-1} b \in S$.

Prendiamo $u, v \in \ker f$. La tesi \`e che $u^{-1} \cdot v \in \ker f$. Quindi $f ( u^{-1} \cdot v ) = 1_{G'}$.
\[
f( u^{-1} \cdot v ) = f(u^{-1}) \ast f(v) = f(u)^{-1} \ast f(v)
\]
Ma $f(u)^{-1} = 1_{G'}$, quindi ho:
\[
f(u)^{-1} \ast f(v) = 1_{G'} \ast 1_{G'} = 1_{G'}
\]

Teorema di omomorfismo per i gruppi

Sia $f : (G, \cdot) \to (G', \ast)$ un morfismo di gruppi, allora $\varepsilon_f$ \`e una congruenza e il gruppo $G / \varepsilon_f$ \`e isomorfo al gruppo $(Im_f, \ast)$. Ogni elemento $[a] \in G / \varepsilon_f$ \`e del tipo $a \cdot \ker f$.
\[
F : (G / \varepsilon_f, \cdot) \to (Im_f, \ast)
\]
\[
\forall [a] \in G / \varepsilon_f [a] = a \cdot \ker f
\]
Dimostrazione:
$b \in [a] \Rightarrow f(a) = f(b)$ per definizione hanno la stessa immagine tramite il morfismo.

Quindi moltiplico entrambi i membri per $f(a)^{-1}: f(a)^{-1} \ast f(b) = 1_{G'} \Rightarrow 1_{G'} = f(a)^{-1} \ast f(b) = f(a^{-1} \cdot b)$

Quindi $u = (a^{-1} \cdot b ) \in \ker f e b = a \cdot u = a \cdot (a^{-1} \cdot b)$

Viceversa, dobbiamo prendere $b \in a \cdot \ker f \Rightarrow b = a \cdot u$. Ha per forza la stessa immagine di $a$, infatti $f(b) = f(a \cdot u) = f(a) \cdot f(u) = f(a) \cdot 1_G = f(a)$. Segue che $b$ \`e nella classe di $a, b \in [a]$.

Non essendo commutativo ho che $b = a \cdot u = v \cdot a$. Posso vedere $b \in a \cdot \ker f$ sia in $\ker f \cdot a$.

La prima \`e la classe laterale sinistra, la seconda \`e la classe laterale destra.
\[
\forall a \in G a \cdot ker f = \ker f \cdot a
\]
\[
b = u \cdot a
\]
\[
f(b) = f(u) \cdot f(a) = 1_{G'} \cdot f(a)
\]
Proiezioni. La seguente \`e la prima proiezione
\[
p_1 : \mathbb{R} \times \mathbb{R} \to \mathbb{R}
\]
\[
p_1 (x, y) = x
\]
\[
p_1 : (\mathbb{R} \times \mathbb{R}, +) \to (\mathbb{R}, +)
\]
Qual \`e l'immagine $Im_{p_1}$? $Im_{p_1} = \{ r \in \mathbb{R} : \exists (x, y) p_1 (x, y) = r \}$. In questo caso l'immagine \`e tutto $\mathbb{R}$. Infatti $\forall r \in \mathbb{R} p_1(r, 0) = r$.

Adesso troviamo il nucleo.
\[
\ker f = \{ (x, y) \in \mathbb{R} \times \mathbb{R} : p_1(x, y) = 0 \} = \{ (0, y) : y \in \mathbb{R} \}
\]
L'unit\`a di $G 1_G = (0, 0) \in \ker p_1$

0 \`e l'elemento neutro di $(\mathbb{R}, +)$.
\[
(2, 3) \in \mathbb{R} \times \mathbb{R}, p_1(2, 3) = 2
\]
\[
[(2, 3)] = \{ (x, y) \in \mathbb{R} \times \mathbb{R} : f(x, y) = 2\}
\]
Applicando il teorema di omomorfismo visto prima vediamo subito che \`e:
\[
[(2,3)] = (2, 3) + \ker p_1 = \{ (2, 3) + (0, y) = (2, y + 3) \}
\]
Altro esempio:
$\mathbb{R}[x] =$ insieme dei polinomi in una indeterminata $x$

un polinomio \`e una espressione formale del tipo $a_0 + a_1 x + \dots a_n x^n$ dove $a_n \neq 0$ e $n$ si dice grado del polinomio.

C'\`e una differenza fra $x \to$ indeterminata e $x \to $ variabile. L'indeterminata fa parte dei polinomi, e vuol dire che $x$ \`e un simbolo. Variabile vuol dire che \`e un elemento di un insieme, ossia $x \in E$. In genere si confonde indeterminata con variabile, perch\'e quando si parla di polinomi la $x$ \`e indeterminata, ma ogni polinomio individua una funzione polinomiale. $p : R \to R a \mapsto p(a)$
\[
p(x) = 1 + 2x
\]
individua la funzione polinomiale $p : R \to R \forall a \in R$ associa $1 + 2 \cdot a = p(a)$. Nel caso dei numeri reali, questa funzione \`e una biezione. Se due polinomi hanno la stessa funzione polinomiale, allora sono lo stesso polinomio. Non \`e vero se prendo altri insiemi.

Per fare i polinomi bisogna avere un campo.

Se prendo $Z_2 = {0,1}$, ossia i resti della divisione per 2. Possiamo definire due operazioni, di somma e di prodotto.

\begin{tabular}{c|cc}
+ & 0 & 1 \\
\hline
0 & 0 & 1 \\
1 & 1 & 0
\end{tabular}

$Z / \equiv_2$, congruenza modulo 2.

\begin{tabular}{c|cc}
$\cdot$ & 0 & 1 \\
\hline
0 & 0 & 0 \\
1 & 0 & 1
\end{tabular}

Un campo \`e un gruppo rispetto a $+$ e un gruppo rispetto a $\cdot$ senza elemento neutro (0).

Posso creare i polinomi a coefficienti in $Z_2$, ossia $Z_2[x]$, ad esempio $1 + x$. La funzione polinomiale rispetto a questo polinomio \`e:
\begin{itemize}
    \item per $x = 0 \to p(0) = 1$
    \item per $x = 1 \to p(1) = 0$
\end{itemize}
Ma dato il polinomio $1 + x^2$ ho:
\begin{itemize}
    \item per $x = 0 \to p(0) = 1$
    \item per $x = 1 \to p(1) = 0$
\end{itemize}
Sono quindi due polinomi diversi che hanno la stessa funzione polinomiale.

Due polinomi sono uguali se hanno tutti i coefficienti uguali.

$(R[x], +)$ \`e un gruppo commutativo. Come funziona il +? Sommo i coefficienti. L'elemento neutro \`e il polinomio nullo, ossia il polinomio con tutti i coefficienti uguali a 0. Si indica con $\underline{0}$. Il polinomio nullo ha grado -1.

Posso definire un morfismo di gruppi su tutta sta merda.

Prendo tutti i polinomi di grado minore o uguale a due, e li indico $R_2[x]$.

Prendo l'applicazione $f : (R_2[x], +) \to (R^2, +)$ definito come segue:
\[
f(a_0 + a_1 \cdot x + a_2 \cdot x^2) = (a_2, a_1 + a_2)
\]
Quindi $1 + 2x \mapsto (0, 2)$.

Qual \`e l'immagine di questa $f$? 
\[
Im_f = \{ (r, s) \in R^2 : \exists p(x) t.c. f(p(x)) = (r,s)\} = R^2
\]
\[
(r, s) = f(0 + (s - r) x + r x^2)
\]
Troviamo il nucleo.
\[
\ker f = \{ p(x) \in R^2[x] : f(p(x)) = (0, 0)\}
\]
Quindi sono tutti i polinomi di grado 0 pi\`u il polinomio nullo, dovendo avere $a_2 = 0$ e $a_1 = 0$.

Tutti i polinomi $[p(x)] = a_0 + a_1 x + a_2 x^2$ devono potersi scrivere come $p(x) + \ker f$.
\[
f(p(x)) = (a_2, a_1 + a_2)
\]
\[
p(x) + \ker f
\]
\[
q(x) \in [p(x)] sono q(x) = p(x) + a_0
\]
con $a_0 \in \ker f$

Esercizio:

$(G, \cdot)$, $(G', \ast)$, poi abbiamo il morfismo $f : (G, \cdot) \to (G', \ast)$
\[
\ker f \subseteq G
\]
\[
Im_f \subseteq G'
\]
Caratterizzano il morfismo, questi due gruppi. f \`e un morfismo iniettivo (monomorfismo) $\Leftrightarrow \ker f = \{ 1_G\}$. Il morfismo \`e suriettivo (epimorfismo) $\Leftrightarrow$ solo se $Im_f = G'$.

Il primo caso \`e evidente, il $\ker f$ ha solo un elemento quindi la classe degli elementi con la stessa immagine $[a] = a \cdot \ker f$ ha un solo elemento, ossia $a$. Si pu\`o anche dimostrare direttamente.

$f$ \`e iniettiva per ipotesi. La tesi \`e che il nucleo \`e costituito da un solo elemento $\ker f = \{ 1_G \} $ ossia $f(1_G) = 1_{G'}$

Viceversa se $\ker f = \{ 1_G \} \Rightarrow f(a) = f(b) \Rightarrow f(a) \cdot f(b)^{-1} = 1_{G'} = f(a \cdot b^{-1}) = 1_{G'} \Rightarrow a \cdot b^{-1} = 1_G \Rightarrow a = b$

Se prendiamo $S$ sottogruppo di $G$ possiamo considerare la classe laterale destra $a S$ e la classe laterale sinistra $S a$. Lo vediamo la prossima volta. 




% SONO ARRIVATO QUI A SCRIVERE


\section{Strutture algebriche con due operazioni}

\begin{description}
    \item[Anelli] un anello \`e una struttura algebrica $(A, +, \cdot)$ t.c. 
    \begin{enumerate}
        \item La prima operazione $\left( A, + \right )$ \`e un gruppo abeliano.
        \item La seconda operazione considerata sull'insieme escluso l'elemento neutro, $(A \setminus \left \{ 0 \right \}, \cdot )$ \`e un semigruppo.
        \item $ \forall a, b, c \in A , \ a \cdot (b + c) = a \cdot b + a \cdot c $
        \item $ \forall a, b, c \in A , \ (a + b) \cdot c = a \cdot c + b \cdot c $
    \end{enumerate}
    \item[Campi] \`e un anello in cui $( A \setminus \left \{ 0 \right \}, \cdot )$ \`e un gruppo abeliano.
\end{description}

Gli interi sono un anello: $\left ( \mathbb{Z}, +, \cdot \right )$.

I razionali sono un campo: $\left ( \mathbb{Q}, +, \cdot \right )$. Anche $\mathbb{R}$ e $\mathbb{C}$ sono un campo.